
\documentclass[11pt]{article}

\usepackage{ifpdf}
\ifpdf 
    \usepackage[pdftex]{graphicx}   % to include graphics
    \pdfcompresslevel=9 
    \usepackage[pdftex,     % sets up hyperref to use pdftex driver
            plainpages=false,   % allows page i and 1 to exist in the same document
            breaklinks=true,    % link texts can be broken at the end of line
            colorlinks=true,
            pdftitle=My Document
            pdfauthor=My Good Self
           ]{hyperref} 
    \usepackage{thumbpdf}
\else 
    \usepackage{graphicx}       % to include graphics
    \usepackage{hyperref}       % to simplify the use of \href
\fi 

\usepackage{listings}
\usepackage{color}

\definecolor{dkgreen}{rgb}{0,0.6,0}
\definecolor{gray}{rgb}{0.5,0.5,0.5}
\definecolor{mauve}{rgb}{0.58,0,0.82}

\lstset{frame=tb,
  language=Java,
  aboveskip=3mm,
  belowskip=3mm,
  showstringspaces=false,
  columns=flexible,
  basicstyle={\small\ttfamily},
  numbers=none,
  numberstyle=\tiny\color{gray},
  keywordstyle=\color{blue},
  commentstyle=\color{dkgreen},
  stringstyle=\color{mauve},
  breaklines=true,
  breakatwhitespace=true,
  tabsize=3
}

\title{Homework 1}
\author{Tyler Angert}
\date{Jan 22 2017}

\begin{document}
\maketitle

\begin{center}
THIS CODE IS MY OWN WORK. IT WAS WRITTEN WITHOUT CONSULTING CODE WRITTEN BY OTHER
STUDENTS OR MATERIALS OTHER THAN THIS SEMESTER'S COURSE MATERIALS. TYLER ANGERT.
\end{center}

\section{ComplexCode 1}
  \begin{lstlisting}
      public void f(int N) {
  		for (int i = 0; i < N; i++) {
  			System.out.println("Hey");

  			if (i == 5) {
  				i = N;
  			}	
  		}
  	}
  \end{lstlisting}

  \subsection{Justification of O(1)}
    \begin{enumerate}
      \item This code features a loop that starts at 0, and increments to N by 1 at each iteration.
      \item When the index reaches the numerical value of 5, i = N. Since the loop ends when i  $<$ N and i is ALWAYS assigned the value of N when it reaches 5, then the loop always executes exactly 5 times.
      \item Therefore it is O(1).
    \end{enumerate}

\section{ComplexCode 4}
  \begin{lstlisting}
    
        public void f(int N) {
  		for (int i = 1; i < N; i *= 2) {
  			System.out.println("Hey");
  			
  			for (int j = 0; j < N; j += 2) {
  				System.out.println("You");
  			}
  		}
  	}
  \end{lstlisting}

  \subsection{Justification of O(NlogN)}
    \begin{enumerate}
      \item The inner loop has complexity of O(N/2) which is O(N). This is because on each iteration of the loop, the index increments by 2, so it covers the entire length of the loop in N/2 operations as opposed to N if it incremented by 1.
      \item The outer loop is O(logN) since on every iteration of the loop, the index multiplies by a factor of 2. Therefore it covers double the distance of the total loop on each iteration. For example, if the input N = 32 and the index starts at 1, then the progression goes: 2, 4, 8, 16, 32, which is 5 steps. Lo and behold, log base 2 of 32 = 5.
      \item Since the inner loop, which is O(N), occurs on each iteration of the outer loop, the total complexity is O(NlogN).
    \end{enumerate}

\section{ComplexCode 9}
  \begin{lstlisting}
    	public int f(int[] a, int N) {
  		if (N <= 0) {
  			return a[0];
  		} else {
  			return a[N-1] + f(a, N-1) + f(a, N-1);
  		}
  	}
  \end{lstlisting}

  \subsection{Justification of O(2\textsuperscript{n})}
    \begin{enumerate}
    \item Each call of the function returns 2 calls to the same function, but with one subtracted iteration. Therefore any single call creates its own subtree of recursive function calls. Since each call to the function produces another 2 instances, each layer of the algorithm thus has a progressively increasing amount of function calls.
    \item At step 1 there is the root call which produces  2 recursive calls. At each of those 2 recursive calls, there are another 2 recursive calls–– so now we have  2 * 2 = 4 recursive calls. At step 3, each child call has another 2 as well, so we have 2 * 2 * 2 calls. It keeps going down, adding another 2 calls at each node until the function returns for good.
    \item Therefore the amount of function calls at any layer N can be represented by 2\textsuperscript{n}. By the time the function has completed for good, there are exactly 2\textsuperscript{n}  operations on the bottom layer of the recursion tree. 
    \end{enumerate}

\section{ComplexCode 14}
  \begin{lstlisting}
   	public void f(int[] a, int N) {
  		HashMap<Integer,Integer> x = new HashMap<Integer,Integer>();
  		
  		for (int i = 0; i < N; i++) {
  			x.put(a[i], 2 * a[i]);
  			x.put(2 * a[i], x.get(a[i]));
  			System.out.println(x.get(2 * a[i]));
  		}
  	}
  \end{lstlisting}

  \subsection{Justification of O(N)}
    \begin{enumerate}
      \item Insertion and retrival in hashmaps are (on average) O(1). However, this operation itself occurs N times in the loop, making it O(N) * O(1) = O(N).
    \end{enumerate}

\section{ComplexCode 20}
  \begin{lstlisting}
    	public int f(int N) {
  		if (N <= 0) {
  			return 1;
  		} else {
  			return 2 * f(N / 2);
  		}
  	}
  \end{lstlisting}

  \subsection{Justification of O(logN)}
    \begin{enumerate}
      \item The base case of this recursive code returns 1 when N $<=$ 0.
      \item Each other call of the function returns 2 times the function with HALF of the previous operations.
      \item Therefore for every function call, a function is returned that only has to complete half as many operations. Therefore by the time we get to the base case, log(N) operations will have been completed. 
    \end{enumerate}

\end{document}  
